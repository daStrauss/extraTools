\documentclass{article}
\usepackage{fullpage}

\title{Some Differential Equations}
\author{David Strauss}

\begin{document}
\maketitle
We are primarily concerned with solving problems that look largely like this:
\begin{equation}
\begin{array}{r l}
\dot{T}_1 & = \frac{k}{N_1} + \frac{\dot{N}}{N} - (T_1 - T_2)\\
\dot{T}_2 & = \frac{k}{N_2} + (T_1 - T_2)\\
\dot{N}_1 &= T_1^{3/2}\exp(\frac{c}{T_1}) N_1 N_2 +
T^{3/2}_1N_3N_1^2\\
\dot{N}_2 &= \dot{N}_1\\
N_2 + N_3 &= C
\end{array}
\end{equation}
The first step will be to do the substitution to reduce this to a standard ODE form: $x' = f(x)$:
\begin{equation}
\begin{array}{r l}
\dot{T}_1 &= \frac{k}{N_1} + T_1^{3/2}\exp(\frac{c}{T_1})N_2 +T_1^{3/2}(C-N_2)N_1\\
\dot{T}_2 &= \frac{k}{N_2} + (T_1-T_2) \\
\dot{N}_1 &=  T_1^{3/2}\exp(\frac{c}{T_1}) N_1 N_2 +
T^{3/2}_1(C-N_2)N_1^2\\
\dot{N}_2 &=  T_1^{3/2}\exp(\frac{c}{T_1}) N_1 N_2 +
T^{3/2}_1(C-N_2)N_1^2\\
\end{array}
\end{equation}

By substituting the rate of change equation for $N$ into the rate equations for $T$ we have a standard ODE. Following that, any of the standard methods can be used:
\begin{enumerate}
\item Explicit Euler
\item Implicit Euler
\item Diagonally Implicit Runge Kutta (DIRK)
\item Implicit Runge Kutta 
\item Multi step
\end{enumerate}

I don't have any intuition as to which of these methods is going to give the \emph{best} performance, but I will do some work to sketch out how these methods work, and a few extensions.

\section{1-Step Euler}
These are well known. You just substitute $y' = \frac{y^{k+1} - y^k}{\Delta t}$ and let $f(y)$ either be given as $f(y^k)$ or $f(y^{k+1})$ for explicit and implicit euler methods respectively. Explicit Euler methods 


\section{Newton Solvers}
With many of the implicit solvers, you often end up needing to find the zeros of a nonlinear equation. Sometimes it is a simple linear system, $Ax-b=0$. You should know how to solve that with all sorts of methods, iterative and direct.

However, even in this trivial example, we have a case where a
nonlinear equation pops up. To solve it, use a newton method. These
can be efficient; Newton Methods give quadratic convergence. If the
Jacobian isn't too difficult to evaluate and invert, then this is
certainly the method of choice. If it is difficult to invert, then
maybe you need to use a Newton-Krylov method, or a Jacobian-Free
Newton Krylov method \cite{Knoll2004}

\bibliographystyle{alpha}
\bibliography{/Users/dstrauss/Documents/bibs/stiff_eqns}


\end{document}